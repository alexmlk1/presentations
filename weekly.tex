\documentclass[%
	a4paper,%							A4 paper
	oneside,%							oneside (left and right margin are equal)  - twoside if I want to print it as a book
	%bibliography=totoc,%	add the bibliography to the table of contents
	%listof=totoc,%				add the list of figures and list of tables to the table of contents
	numbers=noenddot,%		no dot at the end of heading numbers (see the comment below)
	%headsepline,%					line after the page head
	%footsepline,%					line before the page foot
	%headings=small,%			smaller headings
	12pt,%								font size
]{scrreprt}


% input encoding
\usepackage[utf8]{inputenc}

% mathematical symbols
\usepackage{amsmath, amssymb, amsfonts, mathrsfs}

% hyper reference for links or figures
\usepackage{hyperref}

% package to modify itemizations and enumerations
\usepackage{enumitem}

% package for better spacing
\usepackage{xspace}

% package for graphics - the formats png, pdf, jpg are permitted
\usepackage{graphicx}
% default path for figures
\graphicspath{{images/}}

% Theorems
\usepackage{amsthm}
\theoremstyle{definition}
\newtheorem{definition}{Definition}[section]
\newtheorem{definition*}{Definition}
\theoremstyle{remark}
\newtheorem*{remark}{Remark}
\newtheorem*{obs}{Observation}
\newtheorem*{ex}{Example}
\theoremstyle{plain}
\newtheorem{theorem}{Theorem}[section]
\newtheorem{proposition}{Proposition}[section]
\newtheorem{corollary}{Corollary}[section]
\newtheorem{lemma}[theorem]{Lemma}
\renewcommand\qedsymbol{$\blacksquare$}

% define useful mathematical symbols
\def\N{\ensuremath{\mathbb{N}}}
\def\R{\ensuremath{\mathbb{R}}}
\def\X{\ensuremath{\mathcal{X}}}
\def\Y{\ensuremath{\mathcal{Y}}}
\def\G{\ensuremath{\mathcal{G}}}
\def\C{\ensuremath{\mathbb{C}}}
\def\P{\ensuremath{\mathbb{P}}}
\def\Ll{\ensuremath{\mathscr{L}}}
\newcommand{\Ss}{\mathscr{S}}
\newcommand{\Pp}{\mathbb{P}}
\newcommand{\Ee}{\mathbb{E}}
\newcommand{\iid}{\mathop{\sim}\limits^{\text{i.i.d.}}}


\begin{document}

\begin{center}
  \textbf{\LARGE Weekly meetings}
\end{center}
\chapter*{If Influence Functions are the Answer, Then What is the Question?}

\section*{Setting}

Consider a prediction task (regression problem) with:
\begin{itemize}
    \item Input space $\X$;
    \item Output space $\Y$;
    \item Training set $\mathcal{D}^n=\{z_i\}_{i=1}^n$ where $z_i=(x_i,y_i)$ for all $i=1,\dots,n$; 
    \\ $X = (x_1,...,x_n)$, $Y=(y_1,...,y_n)$;
    \item Parameter $\theta \in \Theta := \R^d$;
    \item $f(\theta; x)$ estimator of $\Y\mid \X$;
    \item Loss function $\ell:\Y\times\Y\to\R$ (e.g., 
    $\ell(y',y)\mapsto \|y'-y\|^2$).
\end{itemize}

We aim to minimize the training error (empirical risk):
\[
L(\theta;\mathcal{D}^n) 
    = \frac{1}{n}\sum_{i=1}^n \ell\!\big(f(\theta; x_i),\, y_i\big).
\]

Call 
\[
\hat{\theta} \in \arg\min_{\theta\in\Theta} L(\theta; \mathcal{D}^n).
\]

How different is it from
\[
\hat{\theta}_{\varepsilon, -z}
    = \arg\min_{\theta\in\Theta}
        \bigl( L(\theta; \mathcal{D}^n) - \varepsilon\, \ell(f(\theta; x), y)\bigr)
\]

when $\epsilon=1/n$ and $z=z_{\bar i}$ for some $i\in[n]$?\\
To answer this question, we can re-train the whole model on $\mathcal{D}^n\setminus\{z\}$ (Leave-One-Out method), or use influence functions.\\
In the context where the influence functions are well defined, they are a powerful tool. However, when applied to multi-layer perceptrons, for example, their capability of approximating the effect of the LOO decreases drastically.
This paper presents a new point of view: IFs do not approximate the LOO retraining, but instead the effect of another method they present, called PBRF.

\section*{Influence functions}

\begin{definition*}
    Given $(\bar{x},\bar{y})=\bar{z}\in\mathcal{D}^n$, the \emph{influence loss difference}
    relative to $\bar{z}$ is
    \[
    \mathcal{Q}(\bar{z};\hat\theta)
        = \left.\frac{d}{d\varepsilon}
            \left[L(\hat\theta;\mathcal{D}^n)
            - \varepsilon \,\ell(f(\hat\theta;\bar{x}),\bar{y})\right]
        \right|_{\varepsilon=1/n}.
    \]
\end{definition*}


\textit{Interpretation:}  
It measures how much the training error changes when the data point $\bar{z}$ is removed.

\begin{definition*}
    Given $(\bar{x},\bar{y})=\bar{z}\in\mathcal{D}^n$, the \emph{influence function}
    relative to $\bar{z}$ is
    \[
    \mathcal{I}(\bar{z};\hat\theta)
    = \lim_{\varepsilon\to 0}
        \frac{
        \hat{\theta}_{\varepsilon+1/n,-\bar{z}}
        - \hat{\theta}_{1/n,-\bar{z}}
        }{\varepsilon}.
    \]
\end{definition*}


\textit{Interpretation:}  
It represents the direction in which the optimal parameter moves when the
training objective is perturbed by removing $\bar{z}$.\\
Assuming that $L$ is strongly convex in $\theta$, we can rewrite the previous quantities in the closed forms:
\[
\mathcal{I}(z;\hat\theta)
    =  H_{\hat{\theta}}^{-1}\,
      \nabla \ell(f(\hat{\theta};x),y),
\qquad
\mathcal{Q}(z;\hat\theta)
    = \nabla \ell(f(\hat{\theta};x),y)^{\top}
      H_{\hat{\theta}}^{-1}
      \nabla \ell(f(\hat{\theta};x),y),
\]
where $H_{\hat{\theta}}$ is the Hessian of $L$ at $\hat\theta$.
\begin{remark}
  This part is not really clear: referring to the original reference \cite{cook_residuals_1983}, the derivation should be as follow, with our notation.\\
  Let $L_i(\theta) = (L(\hat\theta;\mathcal{D}^n) - 1/n \,\ell(f(\hat\theta;x_i),y_i))$. Assuming that $L$ is twice differentiable, we can use Taylor near $\hat\theta$:
  \begin{equation*}
    L_i(\theta) = L_i(\hat\theta) + (\theta-\hat\theta)^T L'(\hat\theta) + \frac{1}{2}(\theta-\hat\theta)^T L'(\hat\theta) (\theta-\hat\theta) + \mathrm{O}(\|\theta-\hat\theta\|^3)
  \end{equation*}
  Then, by minimizing the difference $L_i(\theta) - L_i(\hat\theta)$ for $\theta\neq\theta$, we get:
  \begin{equation*}
    \theta = \hat\theta - 2 (HL(\hat\theta))^{-1}JL(\hat\theta).
  \end{equation*}
  From this expression is also clear that the Influence Function estimator represents the parameter after one step of the Newton algorithm starting from $\hat\theta$ trying to get to $\hat\theta_{1/n,-z_i}$. 
\end{remark}

Unfortunately, there is some problems with these formulae:
\begin{enumerate}
    \item the strong convexity is essential. If at a minimum point H has any $0$-eigenvalue, we cannot invert it.
    \item Even when $L$ is strongly convex, the problem is not trivial because computing the inverse of the Hessian and the matrix-vector product are heavy computations.

\end{enumerate}

\textbf{Solution 1.}  
For iHVP, there exist efficient approximations requiring $O(nd)$ flops instead of $O(n^3)$.

\textbf{Solution 2.}  
Change point of view: Influence functions are not approximators of LOO retraining, but instead of the proximal Bregman response function (PBRF).

\section*{Response functions}

We define the response function as:
\[
\hat{r}_z(\varepsilon)
  = \arg\min_{\theta\in\Theta}
      \left( L(\theta;\mathcal{D}^n)
      - \varepsilon\,\ell(f(\theta;x),y) \right).
\]
Observe that $\hat{r}_z(\varepsilon)=\hat{\theta}_{\varepsilon,-z}$ and
$\hat{r}_z(0)=\hat{\theta}$.

Since $\hat{r}$ is differentiable at $0$, we can expand with Taylor at first order and we get:
\[
\hat{r}_{z,\mathrm{lin}}(1/n)
  \approx \hat{\theta}
    + \frac1n H_{\hat{\theta}}^{-1}\,
      \nabla\ell(f(\hat{\theta};x),y).
\]
We require $H_{\hat\theta}$ positive definite to invert it; thus $\hat\theta$ must be a minimizer.\\
To compute influence functions for MLPs, we can approximate $H_{\hat{\theta}}$ using the
Gauss--Newton Hessian (GNH) and add damping:
\[
\mathcal{I}^{\dagger}(z;\hat\theta)
  = \left(
        J_{y,\hat{\theta}}^{\top}
        H_{\ell,\hat{\theta}}
        J_{y,\hat{\theta}}
        + \lambda I
    \right)^{-1}
    \nabla\ell(f(\hat{\theta};x),y),
\]
where $J_{y,\hat{\theta}}$ is the Jacobian of  
$F(\theta) = (f(\theta;x_1),\ldots,f(\theta;x_n))$ and $H_{\ell,\hat{\theta}}$ is the hessian of $\ell(f(\theta;x),y)$ in $\theta=\hat\theta$.\\
Note that the damped GNH is always positive definite, as long as $H_{\hat\theta}$ is SPD.\\
We can get the previous formula by linearizing the response function of the regularized loss:
\[
\begin{aligned}
\hat{r}_{z,\mathrm{damp}}(\varepsilon)
  &= \arg\min_{\theta\in\Theta}
     \Big( L(\theta;\mathcal{D}^n)
      - \varepsilon\,\ell(f(\theta;x),y)
      + \tfrac{\lambda}{2}\|\theta-\hat{\theta}\|^2 \Big),\\[4pt]
\hat{r}_{z,\mathrm{damp,lin}}(1/n)
  &\approx \hat{\theta}+\frac1n\mathcal{I}^{\dagger}(z;\hat\theta).
\end{aligned}
\]
Another issue is that, in practice, the parameter we utilise is not a minimizer of $L$. 
Thus, we want to consider a risk for which the early-stopped point $\theta^s$ is optimal:
\[
\mathcal{L}(\theta;\theta^s,\mathcal{D}^n)
  = \frac1n\sum_{i=1}^n
      D_{\ell^{(i)}}\big(f(\theta;x_i), f(\theta^s;x_i)\big),
\]
with $D_{\ell^{(i)}}$ being the Bregman difference:
\[
D_{\ell^{(i)}}(y,y')
 = \ell(y,y_i)
   - \ell(y',y_i)
   - \nabla_1\ell(y',y_i)^\top(y-y').
\]
Intuitively, this quantity measures the difference between the evaluation of $\ell$ on $y$ and the first order Taylor expansion of $\ell$ around $y'$ computed on $y$.
\begin{obs}
  This quantity is always non-negative as long as $\ell$ is convex.
  To exemplify, choose $\ell(y,y')=\|y-y'\|^2/2$. This yields: 
  $$D_{\ell^{(i)}}(y,y')=\|y-y_i\|^2/2-\|y'-y_i\|^2/2-\langle\nabla \|y'-y_i\|^2/2,y-y_i\rangle=\|y-y'\|^2/2.$$
\end{obs}
\vspace{.5cm}\noindent
Consequently, we can define the \emph{Proximal Bregman Response Function} (PBRF) as:
\[
r^b_{z,\mathrm{damp}}(\varepsilon)
  = \arg\min_{\theta\in\Theta}
      \Big(
      \mathcal{L}(\theta;\theta^s,\mathcal{D}^n)
      - \varepsilon\,\ell(f(\theta;x),y)
      + \tfrac{\lambda}{2}\|\theta-\theta^s\|^2
      \Big).
\]
The interesting property of this object is that the linearized PBRF satisfies
\[
r^b_{z,\mathrm{damp,lin}}(1/n)
  = \theta^s + \frac1n\mathcal{I}^{\dagger}(z;\theta^s).
\]

As a consequence, influence functions are \emph{not} approximating LOO retraining under the original
loss $L$. Instead, they approximate the effect of training from $\theta^s$ under the modified objective
\[
\mathcal{L}(\theta;\theta^s,\mathcal{D}^n)
  - \tfrac1n\ell(f(\theta;x),y)
  + \tfrac{\lambda}{2}\|\theta-\theta^s\|^2.
\]
This fact makes PBRF a more suitable benchmark when testing the performances of IFs.\\
Concrete examples of the fact that PBRF actually achieves good results in the tasks we want are given in the experiment section of the paper.

\section*{Error decomposition}

The approximation error of influence functions decomposes into:

\begin{itemize}
\item \textbf{Warm-start gap:} LOO starts from a random parameter (cold start), while IF are related to $\theta^s$; we can then converge to another "optimal" point;
\item \textbf{Proximity gap:} the factor $||theta-theta^s||$ induces the warm start not to move far away from $theta^s$;
\item \textbf{Non-convergence gap:} in practice we almost never start from a fully trained network;
\item \textbf{Linearization error:} produced by approximating the Taylor expansion at first order;
\item \textbf{Solver error:} inexact iHVP computation.
\end{itemize}
\begin{remark}
    The PBRF formulation eliminates the first three gaps.
\end{remark}

\begin{figure}[hbt]
    \centering
    \includegraphics[width=0.9\linewidth]{Err_deco.png}
    \caption{Visual representation of the error defomposition on different datasets and models. The main focus is that the largest components are the first three.}
\end{figure}

\section*{Influence Functions vs Leverage}
For simplicity, let's consider the Ordinary Least Squares problem.\\
Let $X=(x_1|\dots|x_n)^T\in\R^{(n\times d)}$, $y=(y_1,\dots,y_n)\in\R^n$, $e=(e_1,\dots,e_n)\in\R^n$, and $\theta\in\R^d$.
In this setting, assume the dataset is generated by $y=f(\theta^\star;x)+e=X\,\theta^\star+e$ where $e$ is the observational error. 
We can estimate $\theta^\star$ with $\hat\theta=(X^T X)^{-1}X^T y$. Consequently, the predicted data with our model will be $\hat{y} = X(X^T X)^{-1}X^T y$ and we call $P=X(X^T X)^{-1}X^T$, as it is an orthogonal projection on $\mathrm{ran}(X)$.
\begin{definition*}
    The \emph{leverage score} of the $i$-th sample data is $P_{ii}=x_i(X^T X)^{-1}x_i^T$.
\end{definition*}
This quantity describes how much the $i$-th sample data affects the $i$-th prediction of our model. The bigger it is, the more probable it is that the $i$-th sample point is an outlier.\\
Indeed, if we consider $X=JF(\theta)$ as the $X$ in our case, then the influence loss difference is approximately
\footnote{we can write $HF(\theta)=JF(\theta)^T JF(\theta)+\sum\dots$ . If we omit the second term, we have the sought approximation. This is acceptable when the parameter is close to optimal, since the sum annihilates for optimal parameters.} 
the same as the leverage. However, the interpretation of the problem would become different.\\
In any case, an expression of interest that involves both leverage and influence functions is the following (cf. \cite{leverage}):
\begin{equation*}
    -\mathcal{I}(z;\hat\theta) = \hat\theta- \hat\theta_{1/n, -z} = \frac{(X^TX)^{-1}x_i\hat e_i}{1-P_{ii}},
\end{equation*}
where $\hat{e}_i=y_i-x_i^T\hat\theta$.\\
Observe that the $1-P_{ii}$ at denominator means that outliers also have high influence in the training.
  

\nocite{*}
\bibliographystyle{plain}
\bibliography{refs}

\end{document}